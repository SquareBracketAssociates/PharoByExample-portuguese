% $Author: oscar $
% $Date: 2009-09-18 06:57:20 -0300 (Sex, 18 Set 2009) $
% $Revision: 29170 $

% HISTORY:
% 2006-10-05 - Oscar started
% 2007-05-28 - Stef edit
% 2007-06-06 - Oscar first draft
% 2007-08-14 - Stef corrections
% 2007-09-06 - Lukas review
% 2009-08-12 - Oscar rewrite for Pharo

%=================================================================
\ifx\wholebook\relax\else
% --------------------------------------------
% Lulu:
	\documentclass[a4paper,10pt,twoside]{book}
	\usepackage[
		papersize={6.13in,9.21in},
		hmargin={.75in,.75in},
		vmargin={.75in,1in},
		ignoreheadfoot
	]{geometry}
	\input{../common.tex}
	\pagestyle{headings}
	\setboolean{lulu}{true}
% --------------------------------------------
% A4:
%	\documentclass[a4paper,11pt,twoside]{book}
%	\input{../common.tex}
%	\usepackage{a4wide}
% --------------------------------------------
    \graphicspath{{figures/} {../figures/}}
	\begin{document}
	% \renewcommand{\nnbb}[2]{} % Disable editorial comments
	\sloppy
	\frontmatter
\fi
%=================================================================
\chapter{Prefácio}\chalabel{intro}

%=================================================================
\section*{O que é \pharo?}

\pharo é uma implementação moderna, de fonte aberta, e completa da linguagem de programação e do ambiente \st. \pharo é derivado do \squeak\cite{Inga97a}, uma re-implementação do clássico sistema \st-80. Enquanto o \squeak foi desenvolvido principalmente como uma plataforma para desenvolvimento de software educacional experimental, \pharo esforça-se em oferecer uma plataforma leve e de código aberto para o desenvolvimento de software profissional, e uma plataforma robusta e estável para pesquisa e desenvolvimento em linguagens e ambientes dinâmicos. \pharo serve como a implementação de referência para o arcabouço de desenvolvimento web Seaside.

\pharo resolves some licensing issues with \squeak. Unlike previous versions of \squeak, the \pharo core contains only code that has been contributed under the MIT license. The \pharo project started in March 2008 as a fork of \squeak 3.9, and the first 1.0 beta version was released on July 31, 2009.

Although \pharo removes many packages from \squeak, it also includes numerous features that are optional in \squeak. For example, true type fonts are bundled into \pharo. \pharo also includes support for true block closures. The user interfaces has been simplified and revised.

\pharo is highly portable --- even its virtual machine is written entirely in \st, making it easy to debug, analyze, and change. \pharo is the vehicle for a wide range of innovative projects from multimedia applications and educational platforms to commercial web development environments. 

There is an important aspect behind \pharo: \pharo should not just be a copy of the past but really \emph{reinvent} Smalltalk. Big-bang approaches rarely succeed. \pharo will really favor evolutionary and incremental changes. We want to be able to experiment with important new features or libraries. Evolution means that \pharo accepts mistakes and is not aiming for the next perfect solution in one big step\,---\,even if we would love it. \pharo will favor small incremental changes but a multitude of them. The success of \pharo depends on the contributions of its community.
% The \pharo community will pay attention to your submissions to improve the system.

%=================================================================
\section*{Who should read this book?}

This book is based on \emph{Squeak by Example}\footnote{\sbe}, an open-source introduction to \squeak.
The book has been liberally adapted and revised to reflect the differences between \pharo and \squeak.
This book presents the various aspects of \pharo, starting with the basics, and proceeding to more advanced topics.

This book will not teach you how to program. The reader should have some familiarity with programming languages. Some background with object-oriented programming would be helpful.

This book will introduce the \pharo programming environment, the language and the associated tools.  You will be exposed to common idioms and practices, but the focus is on the technology, not on object-oriented design. Wherever possible, we will show you lots of examples. (We have been inspired by Alec Sharp's excellent book on Smalltalk\cite{Shar97a}.)
\index{Sharp, Alex}

There are numerous other books on \st freely available on the web but none of these focuses specifically on \pharo. See for example:
\url{http://stephane.ducasse.free.fr/FreeBooks.html}

\ifluluelse{}{\newpage} % layout hint
%=================================================================
\section*{A word of advice}

% http://www.surfscranton.com/architecture/KnightsPrinciples.htm

Do not be frustrated by parts of \st that you do not immediately understand.
You do not have to know everything!
Alan Knight expresses this principle as follows\footnote{\url{http://www.surfscranton.com/architecture/KnightsPrinciples.htm}}:
\index{Knight, Alan}
\important{{\bf Try not to care.}
Beginning \st programmers often have trouble because they think they need to understand all the details of how a thing works before they can use it. This means it takes quite a while before they can master \ct{Transcript show: 'Hello World'}. One of the great leaps in OO is to be able to answer the question ``How does this work?'' with ``I don't care''.}

%=================================================================
\section*{An open book}

This book is an open book in the following senses: 

\begin{itemize}

\item	The content of this book is released under the Creative Commons Attribution-ShareAlike (by-sa) license.
		In short, you are allowed to freely share and adapt this book, as long as you respect the conditions of the license available at the following URL: 
		\url{http://creativecommons.org/licenses/by-sa/3.0/}.

\item	This book just describes the core of \pharo.
		Ideally we would like to encourage others to contribute chapters
		on the parts of \pharo that we have not described.
		If you would like to participate in this effort, please
		contact us.  We would like to see this book grow!
\end{itemize}

For more details, visit \pbe.

%=================================================================
\section*{The \pharo community}

The \pharo community is friendly and active.
Here is a short list of resources that you may find useful:

\begin{itemize}
\item \url{http://www.pharo-project.org} is the main web site of \pharo.
%environment built on top of \pharo but whose audience is elementary
%school teachers.) % I remove this [Martial: french contributor]

\item \url{http://www.squeaksource.com} is the equivalent of SourceForge for \pharo projects.
Many optional packages for \pharo live here.
\end{itemize}

%=================================================================
\section*{Examples and exercises}

We make use of two special conventions in this book.

We have tried to provide as many examples as possible.
In particular, there are many examples that show a fragment of code which can be evaluated.  We use the symbol \ct{-->} to indicate the result that you obtain when you select an expression and \menu{print it}:

\begin{code}{@TEST}
3 + 4 --> 7    "if you select 3+4 and 'print it', you will see 7"
\end{code}

In case you want to play in \pharo with these code snippets, you can download a plain text file with all the example code from the book's web site: \pbe.

The second convention that we use is to display the icon \dothisicon{} to indicate when there is something for you to do:

\dothis{Go ahead and read the next chapter!}

%=================================================================
\section*{Acknowledgments}

We would first like to thank the original developers of \squeak for making this amazing \st development environment available as an open source project.

% We would like to thank various people who have contributed to this book.
% In particular, we thank
We would also like to thank Hilaire Fernandes and Serge Stinckwich who allowed us to translate parts of their columns on \st, and Damien Cassou for contributing the chapter on streams.

We especially thank Alexandre Bergel, Orla Greevy, Fabrizio Perin, Lukas Renggli, Jorge Ressia and Erwann Wernli for their detailed reviews.

We thank the University of Bern, Switzerland, for graciously supporting this open-source project and for hosting the web site of this book.

We also thank the Squeak community for their enthusiastic support of this book project, and for informing us of the errors found in the first edition of this book.

%=============================================================
\ifx\wholebook\relax\else
   \bibliographystyle{jurabib}
   \nobibliography{scg}
   \end{document}
\fi
%=============================================================
